\documentclass[main.tex]{subfiles}

\begin{document}

\section{Motion Analysis}
\label{sec:motion-analysis}
The tracking control objective in \eqref{control_obj} is 
achieved if and only if $V^* = 0$, where $V^*$ is defined as
\begin{equation}
  V^* = \frac{1}{2}(E^*-E_r)^2+\frac{1}{2}k_P(l^*-l_r)^2
\end{equation}
\subsection{Convergence of Energy}
\label{subsec:convergence-of-energy}
Consider $E_P \equiv E^*$, $l \equiv l^*$, $\dot{l} \equiv 0$ and
$\dot{l} \equiv 0$ and $u = \ddot{l} \equiv 0$, then, from
\eqref{eq:controller-partial-energy-shaping}
\begin{equation}
  \label{eq:convergence-of-energy}
  (E^*-E_r)(l^*\dot{\theta}^2+g\cos\theta)-k_P(l^*-l_r) \equiv 0
\end{equation}

\noindent \textbf{Case 1:} $E^* = E_r$

In this case, $l^* = l_r$ directly follows from 
\eqref{eq:convergence-of-energy}. The largest invariant set $W$
defined in \eqref{eq:largest-invariant-set} becomes
\begin{equation}
  \label{eq:def-Wr}
  W_r = \left\{ (\theta, l, \dot{\theta}, \dot{l})
    \middle| \frac{1}{2} (l_r \dot{\theta})^2 -
    g l_r \cos\theta \equiv E_r, l \equiv l_r \right\}
\end{equation}
hence, as $t \to \infty$, the closed-loop solution
$(\theta(t), l(t), \dot{\theta}(t), \dot{l}(t))$ achieves the
tracking control objective in \eqref{control_obj}.

\noindent \textbf{Case 2} $E^* \neq E_r$

In this case, from \eqref{eq:convergence-of-energy}
\begin{equation}
  l^* \dot{\theta}^2 + g\cos\theta \equiv 
  \frac{k_P(l^*-l_r)}{E^*-E_r}
\end{equation}
Moreover, since by hypothesis $E_P \equiv E^*$ and $l 
\equiv l^*$
\begin{equation}
  l^* \dot{\theta}^2 - 2g\cos\theta \equiv \frac{2E^*}{l^*}
\end{equation}

Taking difference between the above equations shows that
$\cos\theta$ is a constant, hence $\theta$ is a constant.
Let's denote it as $\theta \equiv \theta^*$, which yields
$\sin\theta^*=0$.

Since $\theta \equiv \theta^*$ constant (hence $\dot{\theta}
= 0$) the two above equations respectively become
\begin{gather}
  \label{eq:convergence-energy:case2:A}
  g\cos\theta^* = \frac{k_P(l^*-l_r)}{E^*-E_r} \\
  E^* = -l^*g\cos\theta^*
\end{gather}

Considering $E_r = -g l_r \cos\theta_{\max}$ from
\eqref{eq:Er-property}, by substituting $E^*$ and $E_r$ into
\eqref{eq:convergence-energy:case2:A} and considering 
$\cos^2\theta^* = 1 - \sin^2\theta^* = 1$ it is possible to
obtain the following
\begin{equation}
  l^* = \frac{l_r(k_P+g^2\cos\theta_{\max}\cos\theta^*)}{k_P+g^2}
\end{equation}
Therefore, since $\sin\theta^*=0$ admits a solution only in
$\{0, \pi\}$, either $(\theta^*, l^*)=(\pi, l_{ue})$ or $(\theta^*,
l^*)=(0, l_{de})$ where
\begin{align}
  l_{ue} = l^*\rvert_{\theta^*=\pi} &=
  \frac{l_r(k_P-g^2\cos\theta_{\max})}{k_P+g^2} \\
  l_{de} = l^*\rvert_{\theta^*=0} &=
  \frac{l_r(k_P+g^2\cos\theta_{\max})}{k_P+g^2}
\end{align}
where ``ue'' and ``de'' respectively denote upright equilibrium
point and downward equilibrium point. To guarantee that the length
of the pendulum $l_{ue}$ and $l_{de}$ are positive in the two
cases respectively assume $k_P>-g^2\cos\theta_{\max}$ and
$k_P>g^2\cos\theta_{\max}$, which can be rewritten as
\begin{equation}
  \label{eq:kP-property}
  k_P > g^2 |\cos\theta_{\max}|
\end{equation}
By considering the above with $0 < \theta_{\max} \le \pi$
in \eqref{eq:Er-property}, it is easy to prove (e.g. by
contradiction) that
\begin{gather}
  0 < l_{ue} \le l_r \\ 0 < l_{de} < l_r
\end{gather}
Let $\Omega$ be the equilibrium point set (which is an
invariant set for the considered system) defined as follows
\begin{equation}
  \label{eq:def-Omega}
  \Omega = \{ (\pi, l_{ue}, 0, 0), (0, l_{de}, 0, 0) \}
\end{equation}

\subsection{Closed-Loop Equilibrium Points}
Consider the two equilibrium points defined in $\Omega$.
Let the state variable vector be $x = (\theta, l, \dot{\theta},
\dot{l})^T$. The state space representation is
\begin{equation}
  \dot{x} = f(x)
\end{equation}
which, considering the dynamics described by \eqref{full_model}
and the controller described by
\eqref{eq:controller-partial-energy-shaping},
can be rewritten as 
\begin{align}
  \dot{x}_1 &= x_3 \\
  \dot{x}_2 &= x_4 \\
  \dot{x}_3 &= -\frac{2 x_3 x_4 + g\sin x_1}{x_2} \\
  \dot{x}_4 &= \frac{(E_P-E_r)(x_2 x_3^2 + g\cos x1)
    -k_P(x_2-l_r)-k_V x_4}{k_D}
\end{align}
where $E_P=x_2^2 x_3^2 / 2 - g x_2 \cos x_1$ from
\eqref{eq:partial-energy}.

By using the indirect method of Lyapunov (also known
as first 
method of Lyapunov) it is possible to study
the stability of the equilibrium points of the system.

The characteristic equation of the Jacobian matrix evaluated
at the upper equilibrium point $x_{ue} = (\pi, l_{ue}, 0, 0)^T$
yields
\begin{equation}
  det(sI-A_{ue}) = \left( s^2 + \frac{k_V}{k_D}s +
    \frac{g^2+k_P}{k_D} \right) \left( s^2 - \frac{g}{l_{ue}}
    \right)
\end{equation}
where $I \in \mathbb{R}^{4 \times 4}$ identity matrix and
\begin{equation}
  A_{ue} = \frac{\partial f(x)}{\partial x} \bigg\rvert_{x=x_{ue}}
\end{equation}
hence, the jacobian $A_{ue}$ has three eigenvalues in the open LHP and
one eigenvalue in the open RHP. Thus, the updward equilibrium point
is unstable and hyperbolic (since all the eigenvalues of $A_{ue}$
have non-zero real parts).
SOMETHING ABOUT TH. 2.10 AND LEBESGUE MEASURE ZERO.

The characteristic equation of the Jacobian matrix evaluated
at the downward equilibrium point $x_{de} = (0, l_{de}, 0, 0)^T$ yields
\begin{equation}
  det(sI-A_{de}) = \left( s^2 + \frac{k_V}{k_D}s +
    \frac{g^2+k_P}{k_D} \right) \left( s^2 + \frac{g}{l_{ue}}
    \right)
\end{equation}
where $I \in \mathbb{R}^{4 \times 4}$ identity matrix and
\begin{equation}
  A_{de} = \frac{\partial f(x)}{\partial x} \bigg\rvert_{x=x_{de}}
\end{equation}
hence, the Jacobian $A_{de}$ two eigenvalues in the open LHP and two
eigenvalues on the imaginary axis (hence, non-hyperbolic since there
is at least one eigenvalue on the imaginary axis). In case of
non-hyperbolic equilibrium points, linearization fails to determine
its stability from the Jacobian matrix. A possible solution
is to use CENTER MANIFOLD THEOREM 2.9 (2.1.4), which is 
however too hard for this system. Consider the Lyapunov
function V defined in
\eqref{eq:lyapunov-function-partial-energy-shaping}.
Consider the following set
\begin{equation}
  \label{eq:Gamma-d-invariant-set}
  \Gamma_d = \left\{ (\theta, l, \dot{\theta}, \dot{l})
    \mid V(\theta, l, \dot{\theta}, \dot{l}) <
    V(0, l_{de}, 0, 0) \right\}
\end{equation}

Let the values of V at the downward equilibrium point $x_{de}$
and a state $(\delta, l_{de}, 0, 0)$ be respectively $V_{de}$
and $V_\delta$
\begin{gather}
  V_{de} = V(0, l_{de}, 0, 0) = \frac{1}{2}(g l_{de} + E_r)^2
    + \frac{1}{2} k_P (l_{de} - l_r)^2 \\
  V_\delta = V(\delta, l_{de}, 0, 0) = \frac{1}{2}(g l_{de}
    \cos\delta + E_r)^2 + \frac{1}{2} k_P (l_{de} - l_r)^2
\end{gather}
Compute the difference between the two above values yields
\begin{align}
  \label{eq:diff-lyapunov-V}
  V_\delta - V_{de} %&= -\frac{1}{2} g l_{de} (1-\cos\delta)
    %(g l_{de}(1+\cos\delta)+2E_r) \\
    &= -\frac{g^2 l_{de} l_r (1-\cos\delta) \Xi}{k_P+g^2}
\end{align}
where
\begin{equation}
  \Xi = k_P(1-\cos\theta_{\max})-(k_P+g^2\cos\theta_{\max})
    \sin^2 \left( \frac{\delta}{2} \right)
\end{equation}
Using $k_P > -g^2\cos\theta_{\max}$ from \eqref{eq:kP-property} and
$|\sin \delta|<|\delta|$ for $\delta \neq 0$ (hence
$k_P+g^2\cos\theta_{\max}>0$ and
$-\sin^2(\delta/2)>-\delta^2/4$)
\begin{equation}
  \Xi > k_P(1-\cos\theta_{\max})-
    \frac{(k_P+g^2\cos\theta_{\max})\delta^2}{4}
\end{equation}
with $\delta \neq 0$. Moreover, if $\delta$ satisfies
\begin{equation}
  \label{eq:delta-property}
  0 < |\delta| \le \delta_m = 2
    \sqrt{\frac{k_P(1-\cos\theta_{\max})}{k_P+g^2\cos\theta_{\max}}}
\end{equation}
then $\Xi>0$ (proof by simply noticing that $-\delta^2\ge-\delta_m^2$),
which, using \eqref{eq:diff-lyapunov-V}, implies
\begin{equation}
  V_\delta < V_{de}, \quad (\delta, l_{de}, 0, 0) \in \Gamma_d
\end{equation}
which implies, because of the
definition of $\Gamma_d$ in \eqref{eq:Gamma-d-invariant-set},
that $\Gamma_d \neq \varnothing$.

Since the considered Lyapunov function
\eqref{eq:lyapunov-function-partial-energy-shaping} is
monotonically decreasing under the controller \eqref{eq:controller-partial-energy-shaping}, for any $\delta$
satisfying \eqref{eq:delta-property}, closed-loop solution
starting from an initial state belonging to $\Gamma_d$
approaches $W_r$ due to the results obtained in
subsection \ref{subsec:convergence-of-energy} (Case 1),
hence, the downward
equilibrium point $(0, l_{de}, 0, 0)$ is unstable in the
Lyapunov sense.

It is, hence, possible to conclude by saying that
every closed-loop solution under the closed-loop
system consisted of the dynamic equation
\eqref{full_model} and the controller
\eqref{eq:controller-partial-energy-shaping}, supposing
$k_P>g^2|\cos\theta_{\max}|$, $k_D>0$, $k_V>0$,
$0<\theta_{\max}\le\pi$, approaches $W=W_r\cup\Omega$,
where $W_r$ is defined in \eqref{eq:def-Wr} and $\Omega$
is defined in \eqref{eq:def-Omega}.

\end{document}