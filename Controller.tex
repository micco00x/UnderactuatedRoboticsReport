\documentclass[main.tex]{subfiles}

\begin{document}

\section{Controller Design}
\label{sec:controller-design}
To propose a design for control input $u$ for the given control tracking problem, in order to achieve the required control objective in equation \eqref{control_obj}, two approaches for the controller design have been considered in \cite{xin2014control}. The first one applies the conventional energy-based control approach, while the second one was proposes modifications for the aforementioned method.
\subsection{Total Energy Shaping}
Consider the conventional Lyapunov candidate
\begin{equation}
    V_c = \frac{1}{2}(E_T-E_r)^2+\frac{1}{2}k_P(l-l_r)^2+\frac{1}{2}k_D\dot{l}^2
\end{equation}
where $k_D,k_P > 0$ are scalar control parameters. Computing the time derivative of $E_T$ defined in equation \eqref{total_energy}
\begin{equation}\label{energy_dot}
    \dot{E}_T = (u-l\dot{\theta}^2-gl\cos\theta)\dot{l}
\end{equation}
then the time derivative of the Lyapunov candidate $V_c$ along the trajectories of \eqref{full_model} is computed as
\begin{equation}
    \dot{V}_c = \dot{l}\big((E_T-E_r+k_D)u-(E_T-E_r)(l\dot{\theta}^2+g\cos\theta)-k_P(l-l_r) \big)
\end{equation}
 
Consider the following controller
\begin{equation}
\label{eq:tot-mech-energy-controller}
    u = \frac{(E_T-E_r)(l\dot{\theta}^2+g\cos\theta)-k_P(l-l_r)-k_V\dot{l}}{E_T-E_r+k_D}
\end{equation}
 for some positive $k_V$, then 
 \begin{equation}
     \dot{V}_c = -k_V\dot{l}^2 \leq 0
 \end{equation}
which holds only if
 \begin{eqnarray}
    E_T-E_r+k_D  \neq 0 \ \ \ \ \  	\forall t\geq 0
 \end{eqnarray}

% Lemma 2.2 part.
% With $E_T \ge -gl(t)$ being the lower bound of the energy of the system, and $l(t)$ being a variable, hence the lower bound.

Given Eq. \eqref{total_energy}, since $E_T \ge -gl(t)$,
the total mechanical energy is not bounded from below.
[THIS BOUNDED FROM BELOW PART MUST BE EXPLAINED
BETTER].
Hence, it is not possible to choose a $k_D$ which would
ensure the proposed controller to be free of singular
points.

\subsection{Partial  Energy Shaping}
Let $E_P$ be the sum of kinetic energy of rotation and potential
energy of the VLP
\begin{equation}
  \label{eq:partial-energy}
  E_P = \frac{1}{2}(l(t)\dot{\theta}(t))^2-gl(t)\cos\theta(t)
\end{equation}

\noindent and consider the following Lyapunov candidate
\begin{equation}
  \label{eq:lyapunov-function-partial-energy-shaping}
  V = \frac{1}{2}(E_P-E_r)^2+\frac{1}{2}k_P(l-l_r)^2+
      \frac{1}{2}k_D\dot{l}^2
\end{equation}

Taking the time derivative of $E_P$ and using the Euler-Lagrange
equation \eqref{full_model} yields
\begin{equation}
  \dot{E}_P = -(l\dot{\theta}^2+g\cos\theta)\dot{l}
\end{equation}

\noindent which, by taking the time derivative of $V$, shows that
\begin{equation}
  \label{eq:partial-energy-shaping:vdot}
  \dot{V} = \big(-(E_P-E_r)(l\dot{\theta}^2+g\cos\theta)+k_P(l-l_r)\big) + k_D u) \dot{l}
\end{equation}

Consider the following controller
\begin{equation}
  \label{eq:controller-partial-energy-shaping}
  u = \frac{(E_P-E_r)(l\dot{\theta}^2+g\cos\theta)-k_P(l-l_r)
      -k_V\dot{l}}{k_D}
\end{equation}

\noindent which is free of singular points. Substituting
\eqref{eq:controller-partial-energy-shaping} back into 
\eqref{eq:partial-energy-shaping:vdot} shows that
\begin{equation}
  \label{eq:partial-energy-shaping:v-nsd}
  \dot{V} = -k_V \dot{l}^2 \le 0, \quad k_V > 0
\end{equation}

Consider the following set
\begin{equation}
  \Gamma_c = \left\{(\theta,l,\dot{\theta},\dot{l})|
      V(\theta,l,\dot{\theta},\dot{l}) \le c\right\},
      \quad c > 0
\end{equation}

Since $\dot{V} \le 0$ from \eqref{eq:partial-energy-shaping:v-nsd},
any closed-loop solution starting in $\Gamma_c$ remains in $\Gamma_c$
for all $t \ge 0$.
Let $W$ be the largest invariant set in
\begin{equation}
    S = \{(\theta,l,\dot{\theta},\dot{l})) \in \Gamma_c|\dot{V}=0\}
\end{equation}

Using LaSalle's invariant principle (Theorem
\ref{th:lasalle-s-invariance-principle}), every closed-loop
solution starting in $\Gamma_c$ approaches $W$ as $t \to
\infty$.

Since $\dot{V} = 0$ holds for all elements of $W$, $V$ and $l$
are constant in $W$ (let them be $V^*$ and $l^*$). Moreover,
since $V$ is a constant in $W$, $E_P$ is also a constant in $W$
(from \eqref{eq:lyapunov-function-partial-energy-shaping}).
Consequently
 \begin{eqnarray}
  \lim_{t\rightarrow \infty} E_P = E^*  \ \ \ \ \ \  \lim_{t\rightarrow \infty} \dot{l} = 0 \ \ \ \ \ \  \ \ \lim_{t\rightarrow \infty} l = l^*
 \end{eqnarray}
 
Thus, the largest invariant set $W$ can be defined as
\begin{equation}
  \label{eq:largest-invariant-set}
  W = \left\{(\theta,l,\dot{\theta},\dot{l})\middle|\frac{1}{2}(l^*\dot{\theta})^2-gl^*\cos\theta \equiv E^*, l \equiv l^*\right\}
\end{equation}

\end{document}