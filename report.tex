%++++++++++++++++++++++++++++++++++++++++
% Don't modify this section unless you know what you're doing!
%\documentclass[letterpaper,12pt]{article}
\documentclass[a4paper]{article}
\usepackage{tabularx} % extra features for tabular environment
\usepackage{amsmath}  % improve math presentation
\usepackage{graphicx} % takes care of graphic including machinery
%\usepackage[margin=1in,letterpaper]{geometry} % decreases margins
\usepackage{cite} % takes care of citations
%\usepackage[final]{hyperref} % adds hyper links inside the generated pdf file
%\hypersetup{
%	colorlinks=true,       % false: boxed links; true: colored links
%	linkcolor=blue,        % color of internal links
%	citecolor=blue,        % color of links to bibliography
%	filecolor=magenta,     % color of file links
%	urlcolor=blue
%}
%++++++++++++++++++++++++++++++++++++++++
\usepackage{indentfirst}
\usepackage{tensor}
\usepackage{amssymb}
\allowdisplaybreaks
\usepackage{bm}
\newcommand{\at}[2][]{#1|_{#2}}
\newcommand\numberthis{\addtocounter{equation}{1}\tag{\theequation}}
\newcommand\norm[1]{\left\lVert#1\right\rVert}

\usepackage[none]{hyphenat} % Avoids to go out of margin

% Fontsize of figure smaller than normalsize:
\usepackage{caption}
\captionsetup[figure]{font=small}
\captionsetup[table]{font=small}

%\usepackage{setspace} % doublespacing
\linespread{1.2}

\begin{document}
\sloppy % Avoids to go out of margin

%\title{Torso Pose Estimation on the\\HRP4 Humanoid Robot\\(draft)}
%\author{Michele Cipriano, Godwin Joey, Lorenzo Vianello\\Supervised by Nicola Scianca}
%\date{\today}
%\maketitle

\title{FP1\\Control of the\\Variable Length Pendulum}								% Title
\author{Michele Cipriano, Karim Ghonim, Khaled Wahba}								% Author
\date{\today}											% Date

\makeatletter
\let\thetitle\@title
\let\theauthor\@author
\let\thedate\@date
\makeatother

\begin{titlepage}
	\centering
    \vspace*{0.5 cm}
    \includegraphics[scale = 0.75]{images/SapienzaLogo}\\[1.0 cm]	% University Logo

    \vspace*{-0.4cm}
    \textsc{\large Department of Computer, Control and\\Management Engineering}\\[2.0 cm]	% Department Name
    \vspace*{1cm}

    { \fontsize{20.74pt}{18.5pt}\selectfont\bfseries \thetitle \par } % title

    \vspace*{0.25cm}
    \textsc{\Large Underactuated Robotics}\\[0.5 cm] % course name

    \vspace*{2.6cm}
	\begin{minipage}{0.4\textwidth}
		\begin{flushleft} \large
			\emph{Professors:}\\
			Leonardo Lanari\\
      Giuseppe Oriolo\\
            %Computer Science Department\\
		\end{flushleft}
	\end{minipage}~
	\begin{minipage}{0.4\textwidth}
		\begin{flushright} \large
			\emph{Students:} \\
			Michele Cipriano\\
            Karim Ghonim\\
            Khaled Wahba\\
            \vspace*{0.2cm}
            %\emph{Supervisor:}\\
			%Nicola Scianca
            %Semester\\
		\end{flushright}
	\end{minipage}\\[3.85 cm]

    \vspace{0.2cm}
    \rule{\linewidth}{0.2 mm} \\[0.3 cm]
    \vspace*{-0.3cm}
    Academic Year 2018/2019
\end{titlepage}

\tableofcontents
\newpage

%\begin{abstract}
%In this experiment we studied a very important physical effect by measuring the
%dependence of a quantity $V$ of the quantity $X$ for two different sample
%temperatures.  Our experimental measurements confirmed the quadratic dependence
%$V = kX^2$ predicted by Someone's first law. The value of the %mystery parameter
%$k = 15.4\pm 0.5$~s was extracted from the fit. This value is
%not consistent with the theoretically predicted $k_{theory}=17.34$~s. We attribute this
%discrepancy to low efficiency of our $V$-detector.
%\end{abstract}

%\begin{abstract}
%Todo.
%\end{abstract}

\section{Introduction}
Ch. 8.1.
Introduction\cite{Xin:2014:CDA:2584545}.

\section{Problem Formulation}
Let's consider a variable length pendulum (VLP) with massless rod and without any
friction. Let $\theta(t)$ be the angle between the pendulum and the y-axis
(counterclockwise), let $l(t)$ be the length of the pendulum, let $m$ be the mass of 
the pendulum and $f(t)$ be the force acting on the mass. Let $(x_G, y_G)$ be
the coordinates of the mass:
\begin{align}
  x_G &= l(t) \text{sin}\theta(t) \\
  y_G &= l(t) \text{cos}\theta(t)
\end{align}

Let $T$ be the kinetic energy of the VLP:
\begin{equation}
  T = \frac{1}{2} m ( \dot{x}_G^2 + \dot{y}_G^2 )
    = \frac{1}{2} m ( l(t) \dot{\theta}^2(t) ) + 
      \frac{1}{2} m ( \dot{l}(t) )^2
\end{equation}

Let $P$ be the potential energy of the VLP:
\begin{equation}
  P = m g y_G = -m g l(t) \text{cos}\theta(t)
\end{equation}

The Lagrangian of the VLP is $L = T - P$. Using Euler-Lagrange equation, it is 
possible to determine the motion equation of the VLP:
\begin{gather}
  \ddot{\theta}(t) + \frac{2 \dot{l}(t) \dot{\theta}(t)}{l(t)} +
    \frac{g \text{sin}\theta(t)}{l(t)} = 0 \\
  m \ddot{l}(t) - m l(t) \dot{\theta}^2(t) - mg\text{cos}\theta(t) = f(t)
\end{gather}

\noindent with $f(t)$ force applied to the mass.

Let's consider $u = \ddot{l}(t)$ as control input for the next sections. Since 
there is no control input controlling directly $\theta(t)$, the system is
underactuated.

Let's assume $m=1$. Let the total mechanical energy of the system $E_T$ be:
\begin{equation}
  E_T = T + P
      = \frac{1}{2}\dot{l}^2(t) + \frac{1}{2}(l(t)\dot{\theta}(t))^2
        - g l(t) \text{cos}\theta(t)
\end{equation}

Let the desired trajectory of swing $E_r$ be:
\begin{equation}
  E_r = \frac{1}{2}(l_r \dot{\theta}(t))^2 - g l_r \text{cos} \theta(t)
\end{equation}

\noindent with $E_r$ desired energy and $l_r$ desired length of the pendulum.
Moreover, being $\theta_{\text{max}} \in (0, \pi]$ maximal angle of the desired swing:
\begin{equation}
  E_r = -g l_r \text{cos}\theta_{\text{max}}
\end{equation}

Given the above assumptions, let's consider the problem of trajectory tracking
control, which consists in determining whether it is possible to design a
control law $u$ such that:
\begin{equation}
  \lim_{t\to\infty} E_T = E_r \qquad
  \lim_{t\to\infty} \dot{l} = 0 \qquad
  \lim_{t\to\infty} l = l_r 
\end{equation}

\section{Total Energy Shaping}
Let's consider the following Lyapunov function:
\begin{equation}
  V_c = \frac{1}{2} (E_T - E_r)^2 +
        \frac{1}{2} k_P (l - l_r)^2 +
        \frac{1}{2} k_D \dot{l}^2
\end{equation}

\noindent with $k_D$ and $k_P$ control parameters.

Proof here.

Controller (8.15):
\begin{equation}
  u = \frac{(E_T - E_r)(l\dot{\theta}^2 + g\text{cos}\theta) - k_P(l-l_r)
    - k_V \dot{l}}{E_T - E_r + k_D}
\end{equation}

Lemma 2.2 not applicable here.

Difficulty of finding a constant $k_D$ satisfying (8.17) for all $t \ge 0$.

\subsection{Experiments}
Ch. 8.5 (singular points in controller 8.18).

\section{Partial Energy Shaping}
Ch. 8.3.2.

\subsection{Motion Analysis}
Ch. 8.4 (Convergence of Energy in 8.4.1, Closed-Loop Equilibrium Points in
8.4.2).

\subsection{Experiments}
Ch. 8.5 (controller 8.23 with initial state $(-\pi/6, 2, 0, 0)$ and
$(-\pi/3, l_{de}, 0, 0)$).

\section{Conclusion}
Ch. 8.6.

%++++++++++++++++++++++++++++++++++++++++
% References section will be created automatically
% with inclusion of "thebibliography" environment
% as it shown below. See text starting with line
% \begin{thebibliography}{99}
% Note: with this approach it is YOUR responsibility to put them in order
% of appearance.

\clearpage
\bibliography{bibliography}
\bibliographystyle{ieeetr}

%\begin{thebibliography}{99}

%\bibitem{melissinos}
%A.~C. Melissinos and J. Napolitano, \textit{Experiments in Modern Physics},
%(Academic Press, New York, 2003).

%\bibitem{Cyr}
%N.\ Cyr, M.\ T$\hat{e}$tu, and M.\ Breton,
% "All-optical microwave frequency standard: a proposal,"
%IEEE Trans.\ Instrum.\ Meas.\ \textbf{42}, 640 (1993).

%\bibitem{Wiki} \emph{Expected value},  available at
%\texttt{http://en.wikipedia.org/wiki/Expected\_value}.

%\end{thebibliography}


\end{document}
