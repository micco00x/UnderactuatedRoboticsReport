\documentclass[main.tex]{subfiles}

\begin{document}

\section{Theoretical Background}
\label{sec:theoretical-background}
\subsection{LaSalle's Invariance Principle}
Let $\dot{x} = f(x)$ be an autonomous system with
$f: D \to \mathbb{R}^n$
locally Lipschitz map from a domain $D \subseteq
\mathbb{R}^n$ into $\mathbb{R}^n$.
\begin{definition}{(Invariant Set)}
A set $W$ is called \textit{invariant set} with respect
to the above autonomous system $\dot{x} = f(x)$ if
\begin{equation}
  x(0) \in W \implies \forall t \in \mathbb{R}:x(t) \in W
\end{equation}
Intuitively, if a solution belongs to $W$ at a certain
instant $t$, then it belongs to $W$ for all instants.
$W$ is called \textit{positively invariant set} if the
above holds $\forall t > 0$.
\end{definition}
\begin{theorem}{(LaSalle's Invariance Principle)}
\label{th:lasalle-s-invariance-principle}
Let $\Gamma \subset D$ be a closed and bounded set (a
compact set) which is invariant with respect to the above
autonomous system $\dot{x} = f(x)$. Let $V: D \to
\mathbb{R}$ be a continuously differentiable function
such that $\dot{V}(x)\le0$ in $\Gamma$. Let
$S = \{x \in \Gamma \mid \dot{V}(x)=0\}$. Let W be the 
largest invariant set in $S$. Then, every solution
starting in $\Gamma$ approaches $W$ as $t \to \infty$.
\end{theorem}

\end{document}